\beginsong{24 lutego\\Bijatyka}[by={sł. Janusz Sikorski; muz oryg. \textit{,,The Twenty Fourth of February''};\break wyk. Stare Dzwony, EKT Gdynia, Trzy Majtki}]
\calcchordswidth{\[C D e]}
\beginverse
\clist{\[G]}To dwudziesty czwarty był lutego
\clist{\[G D]}Poranna zrzedła mgła,
\clist{\[e G]}Z niej wyszło siedem uzbrojonych krypt
\clist{\[C D e]}Turecki niosły znak.
\endverse
\beginchorus\memorize[chorus]
\clist{\[G]}No i znów bijatyka, no i znów bijatyka,
\clist{\[G D]}No bijatyka cały dzień,
\clist{\[e G]}I porąbany dzień, porąbany łeb
\clist{\[C D e]}Razem bracia aż po zmierzch.
\endchorus
\beginverse
\clist{^}Już pierwszy skrada się do burt
\clist{^}A zwie się ,,Goździk'',
\clist{^}Z Algieru Pasza wysłał go
\clist{^}Żeby nam upuścić krwi.
\endverse
\ifphone
\beginchorus\replay[chorus]
\clist{^}No i znów bijatyka, no i znów bijatyka,
\clist{^}No bijatyka cały dzień,
\clist{^}I porąbany dzień, porąbany łeb
\clist{^}Razem bracia aż po zmierzch.
\endchorus
\fi
\beginverse
\clist{^}Już następny zbliża się do burt
\clist{^}A zwie się ,,Róży Pąk'',
\clist{^}Plunęliśmy ze wszystkich luf
\clist{^}Bardzo szybko szedł na dno.
\endverse
\ifphone
\beginchorus\replay[chorus]
\clist{^}No i znów bijatyka, no i znów bijatyka,
\clist{^}No bijatyka cały dzień,
\clist{^}I porąbany dzień, porąbany łeb
\clist{^}Razem bracia aż po zmierzch.
\endchorus
\fi
\beginverse
\clist{^}W naszych rękach dwa i dwa na dnie
\clist{^}Cała reszta zwiała gdzieś,
\clist{^}No a jeden z nich zagnaliśmy
\clist{^}Aż na starej Anglii brzeg.
\endverse
\ifphone
\beginchorus\replay[chorus]
\clist{^}No i znów bijatyka, no i znów bijatyka,
\clist{^}No bijatyka cały dzień,
\clist{^}I porąbany dzień, porąbany łeb
\clist{^}Razem bracia aż po zmierzch.
\endchorus
\fi
\endsong
